\documentclass[final]{beamer}

\usepackage[magyar]{babel}
\usepackage[utf8]{inputenc}
\usepackage[style=alphabetic]{biblatex}
%\AtNextBibliography{\small}
%\renewcommand*{\bibfont}{\small}

\renewcommand*{\bibfont}{\tiny}

\bibliography{refs}

\usepackage[scale=1.3]{beamerposter} 

\usetheme{confposter} 

\setbeamercolor{block title}{fg=ngreen,bg=white} % Colors of the block titles
\setbeamercolor{block body}{fg=black,bg=white} % Colors of the body of blocks
\setbeamercolor{block alerted title}{fg=white,bg=dblue!70} % Colors of the highlighted block titles
\setbeamercolor{block alerted body}{fg=black,bg=dblue!10} % Colors of the body of highlighted blocks
% Many more colors are available for use in beamerthemeconfposter.sty

%-----------------------------------------------------------
% Define the column widths and overall poster size
% To set effective sepwid, onecolwid and twocolwid values, first choose how many columns you want and how much separation you want between columns
% In this template, the separation width chosen is 0.024 of the paper width and a 4-column layout
% onecolwid should therefore be (1-(# of columns+1)*sepwid)/# of columns e.g. (1-(4+1)*0.024)/4 = 0.22
% Set twocolwid to be (2*onecolwid)+sepwid = 0.464
% Set threecolwid to be (3*onecolwid)+2*sepwid = 0.708

\newlength{\sepwid}
\newlength{\onecolwid}
\newlength{\twocolwid}
\newlength{\threecolwid}

%\setlength{\paperwidth}{50in} % A0 width: 46.8in
%\setlength{\paperheight}{38.1in} % A0 height: 33.1in

\setlength{\paperwidth}{48in} % A0 width: 46.8in
\setlength{\paperheight}{36in} % A0 height: 33.1in
\setlength{\sepwid}{0.024\paperwidth} % Separation width (white space) between columns
\setlength{\onecolwid}{0.22\paperwidth} % Width of one column
\setlength{\twocolwid}{0.464\paperwidth} % Width of two columns
\setlength{\threecolwid}{0.708\paperwidth} % Width of three columns
\setlength{\topmargin}{-0.3in} % Reduce the top margin size
%-----------------------------------------------------------

\usepackage{graphicx}  % Required for including images

\usepackage{booktabs} % Top and bottom rules for tables

%https://tex.stackexchange.com/questions/33969/changing-font-size-of-selected-slides-in-beamer
\newcommand\Fontvi{\fontsize{22}{7.2}\selectfont}


%title----------------------------------------------------------------------------------------
%	TITLE SECTION 
%----------------------------------------------------------------------------------------

\title{A comparison of two interaction based random graph models} % Poster title

\author{Cs.Noszály, N.Uzonyi} % Author(s)

\institute{Faculty of Informatics, University of Debrecen} % Institution(s)

%----------------------------------------------------------------------------------------

\begin{document}

\addtobeamertemplate{block end}{}{\vspace*{2ex}} % White space under blocks
\addtobeamertemplate{block alerted end}{}{\vspace*{2ex}} % White space under highlighted (alert) blocks

\setlength{\belowcaptionskip}{2ex} % White space under figures
\setlength\belowdisplayshortskip{2ex} % White space under equations

\begin{frame}[t] % The whole poster is enclosed in one beamer frame

\begin{columns}[t] % The whole poster consists of two major columns, each of which is split into two columns - the [t] option aligns each column's content to the top

\begin{column}{\sepwid}\end{column} % Empty spacer column

%%%%%%% 1 col
\begin{column}{\twocolwid} % Begin a column which is two columns wide (column 1)

%%%%%%% split of 1 col
\begin{columns}[t,totalwidth=\twocolwid] % Split up the two columns wide column

%%%%%%% 1.1 col
\begin{column}{\onecolwid}\vspace{-.6in} % The first column within column 1 (column 1.1)

% the model description
\begin{block}{The N-clique model}\small
Backhausz and Móri in \cite{BaMo} introduced a new class of random graphs 
based on interactions of three vertices. They proved almost sure results 
exhibiting the scale free property. The model and the results were further 
generalized by Fazekas and Porvázsnyik in \cite{FaPo}. Here is a short recipe 
of the generation: Start with an $N$-clique, with weigths one all of its 
subcliques. At each step $N$ vertices interact. W.p. (i.e. with probabilty) 
$p$ add a new vertex $v$ to the graph and select an $N-1$ element set $K$ 
from the old vertices: w.p. $r$ from the $N-1$ cliques proportional to their
 weights or w.p. $1-r$ from the existing vertices uniformly, then set $K=K\cup{v}$. 
 W.p. $1-p$ select an $N$ element set $K$ from the old vertices: w.p. $q$ select 
 from the $N$ cliques proportional to their weights or w.p. $1-q$ from the existing 
 vertices uniformly. In both cases add the $N$-clique generated by $K$ to the graph 
 and increase all of its subcliques weights by one.

\end{block}

% scale-freeness
\begin{block}{Scale-free property}\small
In \cite{BaMo} the authors proved the scale free property in case of $N=3$, for 
the weight of vertices and for the degree, in \cite{FaPo} the same was proved for 
$N\ge 3.$ We generated a few instances of the $N$-clique model for $N=3,4$ with stepsize 
$10^7$. In the following figures the theoretical "power-law" lines are intentionally omited, 
to make easier the pictorial comparison with the $N$-star model.
\vskip 2cm
\begin{figure}
\includegraphics[width=0.8\linewidth]{./fig/klikkdist4v.pdf}
\end{figure}
\end{block}

\begin{block}{Small-world property}\small
Finding the exact diameter of large graphs is a computationally intensive task. 
We implemented the method proposed in \cite{CreMa}. The algorithm iteratively 
refines the lower and
\end{block}

\end{column} % End of column 1.1

%%%%%%% 1.2 col
\begin{column}{\onecolwid}\vspace{-.6in} % The second column within column 1 (column 1.2)
% klikk diameter cont
\begin{block}{}\small
upper bound for the given graph using breadth first searches. 
We generated ca. 1000 instances of the $N$-clique models for various parameters, 
and determined the additive $2$-approximations of the diameter, i.e. an interval 
at most length $2$ consisting the diameter. In the figures the 
red and blue marks are the upper and lower bounds for $\frac{diam(G(V,E))}{\log(|V|)}$. 
From practical point of view one can conclude that the diameter is in $O(\log(|V|))$.
\vskip 2cm
%\begin{center}
\begin{figure}
\includegraphics[width=0.8\linewidth]{./fig/klikkdiam4.pdf}
\end{figure}
%\end{center}
\end{block}


% clustering coefficient
  
\begin{block}{Clustering coefficient}\small
  The local clustering coefficient of a node $v$, is the proportion of the connected 
  vertex pairs and all possible pairs from the adjacency of $v$, see \cite{WS}.
  The exact computation based on counting all triangles in a network, see \cite{Lat}. 
  We implemented a simple $O(|V|maxdeg^{2})$ algorithm. The figures below show that the 
  model has relatively high average clustering coefficient (as the cliques), moreover 
  some kind of convergence can be observed.
  \vskip 2cm
  \begin{figure} 
  \includegraphics[width=0.8\linewidth]{./fig/klikkclust4.pdf}
  \end{figure} 
\end{block}

\end{column}
\end{columns} % End of the split of column 1

%% \begin{column}{\twocolwid}
%% \begin{block}{valami}

%% \end{block}
%% \end{column}

\end{column}%bal oszlop

\begin{column}{\sepwid}\end{column} % Empty spacer column

%%%%%%% 2 col
\begin{column}{\twocolwid} % Begin a column which is two columns wide (column 2)

%%%%%%% split of 2
\begin{columns}[t,totalwidth=\twocolwid] % Split up the two columns wide column

%%%%%%% 2.1 col
\begin{column}{\onecolwid}\vspace{-.6in} % The first column within column 2 (column 2.1)
  
% the model description
\begin{block}{The N-star model}\small
  It is a centralized variant of the $N$-clique model: at each step a 
  (possibly new) vertex and $N-1$ old vertices interact forming an 
  $N$-star, i.e an $N$-tree with a central vertex and $N-1$ leaves.
  The process of the evolution is as follows: Start with an $N$-star, 
  with weigths one all of its substars. At each step $N$ vertices interact. 
  W.p. $p$ add a new vertex $v$ to the graph and select an $N-1$ element set 
  $K$ from the old vertices: w.p. $r$ from the $N-1$ stars proportional to 
  their weights or w.p. $1-r$ from the existing vertices uniformly, then set
   $K=K\cup{v}$. W.p. $1-p$ select $N$ element set $K$ from the old vertices: 
   w.p. $q$ select from the $N$ stars proportional to their weights or w.p. $1-q$ 
   from the existing vertices uniformly.  Note that if no central node in $K$ ($(p,1-r)$ 
   and $(1-p,1-q)$ branches) select it uniformly from the {\small\it old} vertices of $K$. 
   At last add the $N$-star generated by $K$ to the graph and increase all of 
   its substars weights by one.
\end{block}

% scale-freeness
\begin{block}{Scale-free property}\small
There are partial results on the power-law distribution of weights of the vertices:
\cite{Faz}. The weight of a vertex adds up from two parts: peripheral 
and central one. Fixing one of them, convergence proprties can be proved.

\vskip 2cm
  \centering
  \includegraphics[width=0.8\linewidth]{./fig/csilldist4v.pdf}
\end{block}

\begin{block}{Small-world property}\small
Based on the simulation data one can conjecture, that the $N$-star model 
diameter is in $O(\log(|V|)$.
\end{block}


 
\end{column} % End of column 2.1


%%%%%%% 2.2 col
\begin{column}{\onecolwid}\vspace{-.6in} % The second column within column 2 (column 2.2)
% small-world
\begin{block}{}\small
\begin{figure}
  \includegraphics[width=0.8\linewidth]{./fig/csilldiam4.pdf}
\end{figure}
\end{block}


% clustering coefficient
\begin{block}{Clustering coefficient}\small
The model has extremely low clustering cofficient. In the figures the black 
marks represent the average of the observed coefficents for selected stepsizes. 
Log-log scale used, and green line were fitted. 
\vskip 2cm
\begin{figure}
  \includegraphics[width=0.8\linewidth]{./fig/csillclustlog4.pdf}
\end{figure}
\end{block}

\nocite{NU}


\begin{block}{References}
\printbibliography
\end{block}

%----------------------------------------------------------------------------------------

\end{column} % End of column 2.2

\end{columns} % End of the split of column 2 - any content after this will now take up 2 columns width


\end{column} % end of col 2

\end{columns} % End of all the columns in the poster

\end{frame} % End of the enclosing frame

\end{document}
