\begin{block}{The N-clique model}\small

  Backhausz and Móri in \cite{BaMo} introduced a new class of random graphs based on interactions of three vertices. They proved almost sure results exhibiting the scale free property. The model and the results were further generalized by Fazekas and Porvázsnyik in \cite{FaPo}. Here is a short recipe of the generation: Start with an $N$-clique, with weigths one all of its subcliques. At each step $N$ vertices interact. W.p. (i.e. with probabilty) $p$ add a new vertex $v$ to the graph and select an $N-1$ element set $K$ from the old vertices: w.p. $r$ from the $N-1$ cliques proportional to their weights or w.p. $1-r$ from the existing vertices uniformly, then set $K=K\cup{v}$. W.p. $1-p$ select an $N$ element set $K$ from the old vertices: w.p. $q$ select from the $N$ cliques proportional to their weights or w.p. $1-q$ from the existing vertices uniformly. In ether cases add the $N$-clique generated by $K$ to the graph and increase all of its subcliques weights by one.

\end{block}
